\documentclass{article}[12pt]
\usepackage{amsmath,amssymb,authblk,bm,bm,caption,graphicx,hyperref,lscape,mathrsfs,setspace,subcaption,tabularx,tkz-graph,upgreek,url,amsthm}
\usetikzlibrary{shapes.geometric}
\usepackage[letterpaper, margin=2.5cm]{geometry}
\usepackage[backend=biber,style=numeric,isbn=false,sorting=none,giveninits=true,terseinits=true]{biblatex}
\DeclareNameAlias{default}{last-first}
%\doublespacing
\graphicspath{{graphics/}}
\newcommand{\tablefolder}{./TCtables}
\newcommand{\la}{\leftarrow}
\newcommand{\ra}{\rightarrow}
%\usepackage{draftwatermark}
\newcommand{\footremember}[2]{%
	\footnote{#2}
	\newcounter{#1}
	\setcounter{#1}{\value{footnote}}
}
\newcommand{\footrecall}[1]{%
	\footnotemark[\value{#1}]%
}

\newcommand{\absdiv}[1]{%
	\par\addvspace{.5\baselineskip}% adjust to suit
	\noindent\textbf{#1}\quad\ignorespaces
}

\renewcommand*{\revsdnamepunct}{}

\makeatletter
\renewbibmacro*{name:last-first}[4]{%
	\ifuseprefix
	{\usebibmacro{name:delim}{#3#1}%
		\usebibmacro{name:hook}{#3#1}%
		\ifblank{#3}{}{%
			\ifcapital
			{\mkbibnameprefix{\MakeCapital{#3}}\isdot}
			{\mkbibnameprefix{#3}\isdot}%
			\ifpunctmark{'}{}{\bibnamedelimc}}%
		\mkbibnamelast{#1}\isdot
		\ifblank{#4}{}{\bibnamedelimd\mkbibnameaffix{#4}\isdot}%
		%      \ifblank{#2}{}{\addcomma\bibnamedelimd\mkbibnamefirst{#2}\isdot}}% DELETED
		\ifblank{#2}{}{\bibnamedelimd\mkbibnamefirst{#2}\isdot}}% NEW
	{\usebibmacro{name:delim}{#1}%
		\usebibmacro{name:hook}{#1}%
		\mkbibnamelast{#1}\isdot
		\ifblank{#4}{}{\bibnamedelimd\mkbibnameaffix{#4}\isdot}%
		%      \ifblank{#2#3}{}{\addcomma}% DELETED
		\ifblank{#2}{}{\bibnamedelimd\mkbibnamefirst{#2}\isdot}%
		\ifblank{#3}{}{\bibnamedelimd\mkbibnameprefix{#3}\isdot}}}
\makeatother


%\SetWatermarkText{Confidential}
%\SetWatermarkScale{5}
% Sets the default location of pictures\fvset{fontsize=\normalsize} % The font size of all verbatim text can be changed here
%
\DeclareMathOperator{\logit}{logit}
\DeclareMathOperator{\RR}{RR}
\DeclareMathOperator{\OR}{OR}
\DeclareMathOperator{\E}{E}
%
\addbibresource{waning.bib}
\title{Comment on ``The Measurement and Interpretation of Age-Specific Vaccine Efficacy'' (1992)} 
%
\author{Ivo M Foppa}
	%
%
%----------------------------------------------------------------------------------------
\begin{document}
%
\maketitle%
%
In his pioneering 1992 paper, published in this Journal, Farrington investigated the impact of the mode of vaccine action on age-specific estimates of vaccine efficacy. This is very relevant to the current discussion of waning influenza vaccine effectiveness (VE). 
%
Let $R_0 \in [0,1] \subset  \mathbb{R}$ and $\rho,\lambda \in \mathbb{R}_{\ge 0}$.
%
Consider the following expression:
%
\begin{equation}
\label{eq:conv_integral}
F(a) = \int_{0}^{a} g(u) h(a-u) du,
\end{equation}
%
where
\begin{align}
\label{eq:Ga}
G(a) =& 1-(1-R_0) \;e^{-\rho a}\\
\label{eq:Ga_deriv}
G'(a)=&g(a) = \rho (1-R_0) \;e^{-\rho a}
\end{align}
%
and 
\begin{equation}
\label{eq:h}
h(b)=e^{-\lambda b}.
\end{equation} 
%
\paragraph{Question}
What is $F'(a) = \frac{dF}{da}$?
%
\paragraph{Proposed solutions}
\begin{enumerate}
	\item Using the fundamental theorem of calculus, 
	%
	\begin{equation*}
	F'(a)= g(a) h(a-a) = g(a) h(0)
	\end{equation*}
	%
	\item $F(a)$, on the other hand, is a convolution integral; therefore,
	%
	\begin{equation*}
	F(a) = \int_{0}^{a} g(u) h(a-u) du = \int_{0}^{a} g(a-u) h(u) du.
	\end{equation*} 
	%
	Using this in combination with the fundamental theorem of calculus gives 
	%
	\begin{equation*}
	F'(a)= g(a-a) h(a) = g(0) h(a).
	\end{equation*} 
	%
	Usually, however, $g(0) h(a) \neq g(a) h(0)$.
	\item First evaluating $F(a)$ and then taking the derivative with respect to $a$ gives yet another result.
\end{enumerate}
%
%
\end{document} 

