\documentclass{article}[11pt]
\usepackage{amsmath,amssymb,authblk,bm,bm,caption,graphicx,hyperref,lscape,mathrsfs,setspace,subcaption,tabularx,tkz-graph,upgreek,url}
\usetikzlibrary{shapes.geometric}
\usepackage[a4paper, margin=2.5cm]{geometry}
\usepackage[backend=biber,style=numeric,isbn=false,sorting=none,giveninits=true,terseinits=true]{biblatex}
\DeclareNameAlias{default}{last-first}
\doublespacing
\graphicspath{{graphics/}}
\newcommand{\tablefolder}{./TCtables}
\newcommand{\la}{\leftarrow}
\newcommand{\ra}{\rightarrow}
%\usepackage{draftwatermark}
\newcommand{\footremember}[2]{%
	\footnote{#2}
	\newcounter{#1}
	\setcounter{#1}{\value{footnote}}
}
\newcommand{\footrecall}[1]{%
	\footnotemark[\value{#1}]%
}

\newcommand{\absdiv}[1]{%
	\par\addvspace{.5\baselineskip}% adjust to suit
	\noindent\textbf{#1}\quad\ignorespaces
}

\renewcommand*{\revsdnamepunct}{}

\makeatletter
\renewbibmacro*{name:last-first}[4]{%
	\ifuseprefix
	{\usebibmacro{name:delim}{#3#1}%
		\usebibmacro{name:hook}{#3#1}%
		\ifblank{#3}{}{%
			\ifcapital
			{\mkbibnameprefix{\MakeCapital{#3}}\isdot}
			{\mkbibnameprefix{#3}\isdot}%
			\ifpunctmark{'}{}{\bibnamedelimc}}%
		\mkbibnamelast{#1}\isdot
		\ifblank{#4}{}{\bibnamedelimd\mkbibnameaffix{#4}\isdot}%
		%      \ifblank{#2}{}{\addcomma\bibnamedelimd\mkbibnamefirst{#2}\isdot}}% DELETED
		\ifblank{#2}{}{\bibnamedelimd\mkbibnamefirst{#2}\isdot}}% NEW
	{\usebibmacro{name:delim}{#1}%
		\usebibmacro{name:hook}{#1}%
		\mkbibnamelast{#1}\isdot
		\ifblank{#4}{}{\bibnamedelimd\mkbibnameaffix{#4}\isdot}%
		%      \ifblank{#2#3}{}{\addcomma}% DELETED
		\ifblank{#2}{}{\bibnamedelimd\mkbibnamefirst{#2}\isdot}%
		\ifblank{#3}{}{\bibnamedelimd\mkbibnameprefix{#3}\isdot}}}
\makeatother


%\SetWatermarkText{Confidential}
%\SetWatermarkScale{5}
% Sets the default location of pictures\fvset{fontsize=\normalsize} % The font size of all verbatim text can be changed here
%
\DeclareMathOperator{\logit}{logit}
\DeclareMathOperator{\RR}{RR}
\DeclareMathOperator{\OR}{OR}
\DeclareMathOperator{\E}{E}
%
\addbibresource{waning.bib}
\title{Appearance of Waning Immunity in Studies of Influenza Vaccine Effectiveness due to Bias} 
%
\author[1,2,*]{Ivo M Foppa}
	%
\author[2]{Jill Ferdinands} 
\affil[1]{Battelle Memorial Institute, Atlanta, Georgia, USA}
\affil[2]{Influenza Division, Centers for Disease Control and Prevention, 1600 Clifton Road NE, Atlanta, 30333 Georgia, USA}
\affil[*]{Corresponding Author, Influenza Division, Centers for Disease Control and Prevention, 1600 Clifton Road NE, MS A-20, Atlanta, 30333 Georgia, USA, \nolinkurl{vor1@cdc.gov}}
\date{}
%
%----------------------------------------------------------------------------------------
\begin{document}
	
\maketitle%
%
\clearpage
%
\clearpage
%
\begin{abstract}
\ldots
\end{abstract}
\clearpage
%
%
\clearpage
\section*{Introduction} 
In his enlightening editorial on the challenges faced by observational vaccine effectiveness (VE) studies, \textcite{Lipsitch2018challenges} spells out two mechanisms causing apparent waning in ``leaky'' vaccines: First, heterogeneous risk of infection will deplete the population of those with higher risk first, among the vaccinated slower than the unvaccinated \cite{Margheri2017heterogeneity}.  This leads to relative increase in vaccinated case, resulting in lower VE estimates over time.  The second mechanism, due to incomplete case ascertainment \cite{Wu2018influence}, is not further discussed by \citeauthor{Lipsitch2018challenges}. Here, we reconsider that mechanism in the context of observational influenza VE studies. Presently, most influenza VE studies are test-negative studies (TNS) 

If, hypothetically, 

uses the paper by Ray et al. \cite{Ray2018Intra-season} to highlight some of these challenges. 
%
\section*{Acknowledgements}
\ldots
\clearpage
%
\printbibliography
%
%
%
\end{document} 

