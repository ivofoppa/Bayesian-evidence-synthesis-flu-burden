\documentclass{article}[11pt]
\usepackage{amsmath,amssymb,authblk,bm,bm,caption,graphicx,hyperref,lscape,mathrsfs,setspace,subcaption,tabularx,tkz-graph,upgreek,url,amsthm}
\usetikzlibrary{shapes.geometric}
\usepackage[a4paper, margin=2.5cm]{geometry}
\usepackage[backend=biber,style=numeric,isbn=false,sorting=none,giveninits=true,terseinits=true]{biblatex}
\DeclareNameAlias{default}{last-first}
\doublespacing
\graphicspath{{graphics/}}
\newcommand{\tablefolder}{./TCtables}
\newcommand{\la}{\leftarrow}
\newcommand{\ra}{\rightarrow}
%\usepackage{draftwatermark}
\newcommand{\footremember}[2]{%
	\footnote{#2}
	\newcounter{#1}
	\setcounter{#1}{\value{footnote}}
}
\newcommand{\footrecall}[1]{%
	\footnotemark[\value{#1}]%
}

\newcommand{\absdiv}[1]{%
	\par\addvspace{.5\baselineskip}% adjust to suit
	\noindent\textbf{#1}\quad\ignorespaces
}

\renewcommand*{\revsdnamepunct}{}

\makeatletter
\renewbibmacro*{name:last-first}[4]{%
	\ifuseprefix
	{\usebibmacro{name:delim}{#3#1}%
		\usebibmacro{name:hook}{#3#1}%
		\ifblank{#3}{}{%
			\ifcapital
			{\mkbibnameprefix{\MakeCapital{#3}}\isdot}
			{\mkbibnameprefix{#3}\isdot}%
			\ifpunctmark{'}{}{\bibnamedelimc}}%
		\mkbibnamelast{#1}\isdot
		\ifblank{#4}{}{\bibnamedelimd\mkbibnameaffix{#4}\isdot}%
		%      \ifblank{#2}{}{\addcomma\bibnamedelimd\mkbibnamefirst{#2}\isdot}}% DELETED
		\ifblank{#2}{}{\bibnamedelimd\mkbibnamefirst{#2}\isdot}}% NEW
	{\usebibmacro{name:delim}{#1}%
		\usebibmacro{name:hook}{#1}%
		\mkbibnamelast{#1}\isdot
		\ifblank{#4}{}{\bibnamedelimd\mkbibnameaffix{#4}\isdot}%
		%      \ifblank{#2#3}{}{\addcomma}% DELETED
		\ifblank{#2}{}{\bibnamedelimd\mkbibnamefirst{#2}\isdot}%
		\ifblank{#3}{}{\bibnamedelimd\mkbibnameprefix{#3}\isdot}}}
\makeatother


%\SetWatermarkText{Confidential}
%\SetWatermarkScale{5}
% Sets the default location of pictures\fvset{fontsize=\normalsize} % The font size of all verbatim text can be changed here
%
\DeclareMathOperator{\logit}{logit}
\DeclareMathOperator{\RR}{RR}
\DeclareMathOperator{\OR}{OR}
\DeclareMathOperator{\E}{E}
%
\addbibresource{waning.bib}
\title{Appearance of Waning Immunity in Studies of Influenza Vaccine Effectiveness due to Bias} 
%
\author[1,2,*]{Ivo M Foppa}
	%
\author[2]{Jill Ferdinands} 
\affil[1]{Battelle Memorial Institute, Atlanta, Georgia, USA}
\affil[2]{Influenza Division, Centers for Disease Control and Prevention, 1600 Clifton Road NE, Atlanta, 30333 Georgia, USA}
\affil[*]{Corresponding Author, Influenza Division, Centers for Disease Control and Prevention, 1600 Clifton Road NE, MS A-20, Atlanta, 30333 Georgia, USA, \nolinkurl{vor1@cdc.gov}}
\date{}
%
%----------------------------------------------------------------------------------------
\begin{document}
	
\maketitle%
%
\clearpage
%
\clearpage
%
\begin{abstract}
\ldots
\end{abstract}
\clearpage
%
%
\clearpage
\section*{Introduction} 
In his enlightening editorial on the challenges faced by observational vaccine effectiveness (VE) studies, \textcite{Lipsitch2018challenges} spells out two mechanisms causing apparent waning in ``leaky'' vaccines: First, heterogeneous risk of infection will deplete the population of those with higher risk first, among the vaccinated slower than the unvaccinated \cite{Margheri2017heterogeneity}.  This leads to relative increase in vaccinated case, resulting in lower VE estimates over time.  The second mechanism, due to incomplete case ascertainment \cite{Wu2018influence}, is not further discussed by \citeauthor{Lipsitch2018challenges}. Here, we reconsider that mechanism in the context of observational influenza VE studies. Presently, most influenza VE studies are test-negative studies (TNS) and investigate its impact on VE estimates. We also comment on the recent manuscript by Ray et al. \cite{Ray2018Intra-season} to highlight some of these challenges. 
%
\section*{Notation and theoretical considerations}
Assume that, for the sake of the argument, we conduct a TNS for the assessment of influenza VE, with full ascertainment of all influenza infections, both symptomatic and asymptomatic. Using the notation of \textcite{Wu2018influence}, $\beta \in [0,1]$ is the probability infection due of a contact which, in the unvaccinated, would have resulted in infection. If the rate of influenza infection in the unvaccinated were to be $\lambda_I$, the rate in the vaccinated would amount to $\beta \lambda_I$ and the rate ratio of infection, comparing vaccinated to unvaccinated, would be $\beta$. It is well known that the incidence rate ratio $\beta$ is estimated by the odds ratio, as long as incidence density sampling is followed and controls are drawn from ``only actual candidates for the
illness''\cite{Miettinen1976estimability}, i.e. people susceptible to influenza infection. 

Further assume that all previous infections in the given season can be identified, e.g. serologically, and that pre-existing immunity does not differ by immunity and that vaccination is only administered before the start of seasonal influenza transmission. VE can then be estimated as
%
\begin{equation}
\label{eq:VE_estimate}
1- \hat{\beta}(\tau) = 1 - \frac{c_{11}(\tau) \; c_{00}(\tau)}{c_{10}(\tau)\; c_{01}(\tau)},
\end{equation}
%
where $c_{11}(\tau)$ and $c_{10}(\tau)$ represent the cumulative vaccinated and unvaccinated cases, i.e. influenza infections, up to day $\tau$ and $c_{01}$ and $c_{00}$ represent vaccinated and unvaccinated controls, respectively. Let controls be all subjects infected with a non-influenza respiratory virus which is unaffected by influenza vaccination, and $\lambda_I(t)$ is the incidence rate of influenza infection in the unvaccinated on day $t$ and $\lambda_{nI}(t)$ is the incidence rates of other respiratory viral infections and $\Lambda_I(t) = \int_{u=0}^{t} \lambda_I(t) dt$; $N$ is the total population and $\nu$ is vaccination uptake in that population.
If influenza infection is stochastically independent of alternate respiratory virus infections, then 
\begin{equation}
\label{eq:unbiased}
\E(\hat{\beta}) = \beta,
\end{equation}
%
i.e. \eqref{eq:VE_estimate} is an unbiased estimator of VE because, using the basic rules of algebra, expectations and integration,
%

\begin{proof}
	\begin{align}
\E(1- \hat{\beta}(\tau)) &= 1 - \E\big(\frac{c_{11}(\tau) \; c_{00}(\tau)}{c_{10}(\tau)\; c_{01}(\tau)}\big)\\
  &= 1 - \frac{\E(c_{11}(\tau)) \; \E(c_{00}(\tau))}{\E(c_{10}(\tau))\; \E(c_{01}(\tau))} \\
  &= 1 - \frac{N \; \nu \; \int_{t=0}^{\tau} e^{-\beta \Lambda_I(t)} \; \beta \lambda_I(t) dt \times 
  	N \; (1 - \nu) \; \int_{t=0}^{\tau} e^{-\Lambda_I(t)}\; \lambda_{nI}(t) dt}{
  		N \; (1 - \nu) \; \int_{t=0}^{\tau} e^{-\Lambda_I(t)}\; \lambda_I(t) dt \times
  		N \; \nu \; \int_{t=0}^{\tau} e^{-\beta \Lambda_I(t)}\;\lambda_{nI}(t) dt} \\
  	&= 1 - \frac{\big(1 - e^{ - \beta \Lambda_I(\tau)}\big)\times \big(1 - e^{ - \Lambda_I(\tau)}\big)}{\big(1 - e^{ - \Lambda_I(\tau)}\big)\times \frac{1}{\beta} \; \big(1 - e^{ - \beta \Lambda_I(\tau)}\big) }\\
  			&= \beta 
\end{align}
%
\end{proof}
%
Full ascertainment of all infections, however, is hardly possible. Therefore, \eqref{eq:VE_estimate} will not be an unbiased estimate of VE, i.e.
%
\begin{align}
\E (1 - \hat{\beta}(\tau)) & = 1 - \frac{\big(1 - e^{ - \beta \Lambda_I(\tau)}\big)}{\big(1 - e^{ - \Lambda_I(\tau)}\big)} \\
 &\neq \beta.
\end{align}
%
In addition, relevant to this discussion, q
 more
%
\section*{Acknowledgements}
\ldots
\clearpage
%
\printbibliography
%
%
%
\end{document} 

